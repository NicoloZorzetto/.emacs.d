% Created 2021-01-27 mer 20:20
% Intended LaTeX compiler: pdflatex
\documentclass[11pt]{article}
\usepackage[utf8]{inputenc}
\usepackage[T1]{fontenc}
\usepackage{graphicx}
\usepackage{grffile}
\usepackage{longtable}
\usepackage{wrapfig}
\usepackage{rotating}
\usepackage[normalem]{ulem}
\usepackage{amsmath}
\usepackage{textcomp}
\usepackage{amssymb}
\usepackage{capt-of}
\usepackage{hyperref}
\author{Nicolò Zorzetto}
\date{\today}
\title{Emacs Config}
\hypersetup{
 pdfauthor={Nicolò Zorzetto},
 pdftitle={Emacs Config},
 pdfkeywords={},
 pdfsubject={},
 pdfcreator={Emacs 27.1 (Org mode 9.3)}, 
 pdflang={English}}
\begin{document}

\maketitle
\tableofcontents

:tangle yes

\section{Packages}
\label{sec:orgc1ecc4d}
\subsection{Add reos and initialize.}
\label{sec:orgbde2437}
Adding these 2 repos seems to solve package-install problems in MacOS. Maybe it's only me, but leaving them will not cause problems nor slow down the startup by a lot.
Melpa is the default one and adding the gnu repo is a bit unorthodox but in Fedora just the first didn't list all packages available so I had to add the latter.
\begin{verbatim}
(setq package-archives '(("melpa" . "https://melpa.org/packages/")
			 ("gnu" . "https://elpa.gnu.org/packages/")))
(package-initialize)
\end{verbatim}

\subsection{Use package}
\label{sec:orgdcadca0}
The use-package macro allows you to isolate package configuration in your .emacs file in a way that is both performance-oriented and, well, tidy. I created it because I have over 80 packages that I use in Emacs, and things were getting difficult to manage. Yet with this utility my total load time is around 2 seconds, with no loss of functionality!
\begin{verbatim}
(package-install 'use-package)
\end{verbatim}

\subsection{Elpy}
\label{sec:orgdd6872d}
Elpy is a package that renders Emacs almost a Python IDE.
\begin{verbatim}
  (use-package elpy
  ;; :config
  ;; ;; Hook to start elpy when in python mode (visiting a python file or buffer)
  ;; (add-hook 'python-mode-hook (lambda ()
  ;; (elpy-enable)))
  ;; ;; Set Ipython's envoirment
  ;; ;; (setq python-shell-interpreter "jupyter"
  ;; ;; python-shell-interpreter-args "console --simple-prompt"
  ;; ;; python-shell-prompt-detect-failure-warning nil)
  ;; (add-to-list 'python-shell-completion-native-disabled-interpreters
  ;; "jupyter")
  ;; (elpy-enable))
)
\end{verbatim}

\subsection{Dashboard}
\label{sec:orgdce6821}
The initial buffer.
\begin{verbatim}
;; (use-package dashboard
;; :ensure t
;; :config
;; (setq dashboard-banner-logo-title "Welcome back, Nicolò")
;; (setq dashboard-center-content t)
;; (setq dashboard-set-heading-icons t)
;; (setq dashboard-set-file-icons t)
;; (setq dashboard-set-footer nil)
;; (dashboard-setup-startup-hook))
\end{verbatim}
\subsubsection{Page break lines}
\label{sec:orge4045a6}

\begin{verbatim}
(use-package page-break-lines
:ensure t
:config
(turn-on-page-break-lines-mode))
\end{verbatim}
\subsection{Magit}
\label{sec:orgd48d6ed}

\begin{verbatim}
(use-package magit
:config
(global-set-key (kbd "C-g") 'magit))
\end{verbatim}

\subsection{Ivy}
\label{sec:org4e4f5fd}
\subsubsection{Ivy}
\label{sec:org0460c3d}
Ivy is an interactive interface for completion in Emacs. Emacs uses completion mechanism in a variety of contexts: code, menus, commands, variables, functions, etc. Completion entails listing, sorting, filtering, previewing, and applying actions on selected items. When active, ivy-mode completes the selection process by narrowing available choices while previewing in the minibuffer. Selecting the final candidate is either through simple keyboard character inputs or through powerful regular expressions.
\href{https://github.com/Microsoft/ivy}{Ivy github repo]
[[https://oremacs.com/swiper/][Ivy official documentation}
\begin{verbatim}
(use-package ivy
:config
;; This allows minibuffer commands while inside the minibuffer. I believe my Ivy settings use it quite a bit.
(setq enable-recursive-minibuffers t)
;; This adds bookmarks and recent files to 'ivy-switch-buffer'. This adds a bit of initial load time but is pretty useful in my opinion.
(setq ivy-use-virtual-buffers t)
;; Counsel-minibuffer-history will show the minibuffer commands history. The keybinding only works when inside the minibuffer.
(define-key minibuffer-local-map (kbd "C-r") 'counsel-minibuffer-history)
;; Consel-M-x is Ivy's version of 'execute-extended-command'. 
(global-set-key (kbd "M-x") 'counsel-M-x)
;; Counsel-find-file add Ivy completion to find-file.
(global-set-key (kbd "C-x C-f") 'counsel-find-file)
;; Consel-find-library opens the minibuffer listing all available libraries (often associated with packages) and lets you search and select one opening it's source code.
(global-set-key (kbd "<f1> l") 'counsel-find-library)
;; Consel-describe-function opens the minibuffer listing all available functions and opens the documentation page for the one you select.
(global-set-key (kbd "<f1> f") 'counsel-describe-function)
;; Consel-describe-variable opens the minibuffer listing all available variables and opens the documentation page for the one you select.
(global-set-key (kbd "<f1> v") 'counsel-describe-variable)
;; Consel-describe-symbol opens the minibuffer listing all available symbols and opens the documentation page for the one you select.
(global-set-key (kbd "<f1> o") 'counsel-describe-symbol)
;; Ivy-resume allows you to get back where you left the last ivy completion.
(global-set-key (kbd "<f6>") 'ivy-resume)
:init
;; This activates Ivy
(ivy-mode 1))
\end{verbatim}

\subsubsection{Counsel}
\label{sec:org86f90c1}
Consuel is a collection of Ivy enhanced emacs commands.
\href{https://github.com/Microsoft/ivy}{Ivy github repo}
\href{https://oremacs.com/swiper/}{Ivy official documentation}
\begin{verbatim}
(package-install 'counsel)
\end{verbatim}

\subsubsection{Swiper}
\label{sec:org50c6d66}
Swiper is an an Ivy enhanced Isearch replacement.
\href{https://github.com/Microsoft/ivy}{Ivy github repo}
\href{https://oremacs.com/swiper/}{Ivy official documentation}
\begin{verbatim}
(use-package swiper
:config
(global-set-key (kbd "C-s") 'swiper))
\end{verbatim}
\subsection{Writeromm}
\label{sec:org5343520}
I have long wanted a sort of 'zen-mode' for emacs to integrate in my workflow so to have a super minimal and distraction free envoirment to relax while writing or sometimes coding.
writeroom-mode adds just that to Emacs, emulating the famous OSX editor of the same name. It is not perfect but I believe I can make it more so thru configuration and hooks.
The repo doesn't appear to be the original but seems to still get updates since the other's last commit is from 2015.
\href{https://github.com/joostkremers/writeroom-mode}{Writeroom-mode github repo}
\subsubsection{Installation}
\label{sec:orge4a5a54}
\begin{verbatim}
(use-package writeroom-mode
:config
(global-set-key (kbd "C-x C-w") 'writeroom-mode))
\end{verbatim}

\subsection{PDF tools}
\label{sec:org4647413}
PDF-tools is a package that let's you view PDF files (and more) inside Emacs.
Midnight mode actively changes the colors of PDFs to be homougenous with the theme you have loaded. 
I added a hook to activate it automatically since I use pdf-tools mainly to read books or documents. If you need to see the original colors just disable it manually with < \emph{C-c C-r m} >.
\begin{verbatim}
(use-package pdf-tools
:config
(add-hook 'PDFView-mode-hook (lambda ()
				(pdf-view-midnight-minor-mode)))
(pdf-tools-install))
\end{verbatim}

\subsection{EMMS}
\label{sec:orge1e41c8}
EMMS is a super minimalistic music player inside emacs. It is standardly launched with the command < \emph{M-x RET emms RET} >.
\begin{verbatim}
(package-install 'emms)

(use-package emms
:config 
;;Midnight mode actively changes the colors of PDFs to be homougenous with the theme you have loaded. I added a hook to activate it automatically since I use pdf-tools mainly to read books or documents. If you need to see the original colors just disable it manually with < /C-c C-r m/ >.
(global-set-key (kbd "C-c C-x C-e") 'emms)
;; I have binded /add directory/ to < /C-c C-+/ >. This function opens the minubuffer prompting for a directory and add to emms all music files contained in it.
(global-set-key (kbd "C-c C-+") 'emms-add-directory)
;; Player config
(setq exec-path (append exec-path '("/usr/local/bin")))
(add-to-list 'load-path "~/.emacs.d/site-lisp/emms/lisp")
(require 'emms-setup)
(require 'emms-player-mplayer)
(emms-standard)
(emms-default-players)
(define-emms-simple-player mplayer '(file url)
      (regexp-opt '(".ogg" ".mp3" ".wav" ".mpg" ".mpeg" ".wmv" ".wma"
		    ".mov" ".avi" ".divx" ".ogm" ".asf" ".mkv" "http://" "mms://"
		    ".rm" ".rmvb" ".mp4" ".flac" ".vob" ".m4a" ".flv" ".ogv" ".pls"))
      "mplayer" "-slave" "-quiet" "-really-quiet" "-fullscreen"))
\end{verbatim}
\section{Visual settings}
\label{sec:org7ff4c18}
\subsection{Hide the menubar}
\label{sec:org6b3dfb9}
\begin{verbatim}
(menu-bar-mode 0)
\end{verbatim}

\subsection{Hide the toolbar}
\label{sec:org5581913}
\begin{verbatim}
(tool-bar-mode 0)
\end{verbatim}

\subsection{Show line numbers (kbd toggle)}
\label{sec:org5a5afc6}
Line numbers are not shown by default so I added a toggle keybinding to show/hide them with < \emph{C-x C-l} >
\begin{verbatim}
(global-set-key (kbd "C-x C-l") 'global-display-line-numbers-mode)
\end{verbatim}

\subsection{Visual line mode}
\label{sec:org469a493}
Visual line mode makes line-dependent commands act on visual lines instead of logical ones (so separating wrapped lines). 
\begin{verbatim}
(global-visual-line-mode 1)
\end{verbatim}

\subsection{[theme] spacemacs-dark}
\label{sec:org868567c}
Simple and pretty minimal theme that works well with my DWM config.
\begin{verbatim}
(load-theme 'poet-dark t)
\end{verbatim}

\section{Org-mode settings}
\label{sec:org07508a5}
\subsection{Agenda settings}
\label{sec:org597b0dc}
\subsubsection{Set agenda files}
\label{sec:orgfd3857c}
\begin{verbatim}
(require 'org)
(setq org-agenda-files (list "~/Documents/org/agenda.org"))
\end{verbatim}

\subsubsection{Agenda KBD}
\label{sec:org36672bf}
\begin{verbatim}
(global-set-key (kbd "C-x C-a") 'org-agenda)
\end{verbatim}

\section{Keybindings specific to Italian Keyboards}
\label{sec:org4c55aa7}
Using the Italian keyboard layout and emacs can be troublesome. 
In GNU/Linux systems these are taken care of by default, with the exception of the tidle ("\textasciitilde{}"),  but in MacOS and Windows(?) they are noy. 
Leaving these even if you run GNU/Linux should not cause any problem or warning.
If you use another layout simply put your curson on the "* Keybindings specific to Italian Keyboards" line and press "C-c C-x C-w" to delete the whole entry. This also works with any entry in any org-document.
\href{https://www.gnu.org/software/emacs/manual/html\_node/elisp/Key-Binding-Commands.html}{Manual article on keybindings}
\subsection{Insert "\textasciitilde{}"}
\label{sec:org34c17c1}
On italian keyboards the tidle ("\textasciitilde{}") is inserted with 'Alt+5'. This command makes it possible to do so.
\begin{verbatim}
(global-set-key (kbd "M-5") "~")
\end{verbatim}
\subsection{Insert "\#"}
\label{sec:org11cb40a}
On italian keyboards the pound sign, also called octothorpe or hashtag, ("\#") is inserted with 'Alt+à'. This command makes it possible to do so.
\begin{verbatim}
(global-set-key (kbd "M-à") "#")
\end{verbatim}
\subsection{Insert "[" and "]"}
\label{sec:orgb3ef9e5}
On italian keyboards the square parentheses are inserted with 'Alt+è' or 'Alt++'. This command makes it possible to do so.
\begin{verbatim}
(global-set-key (kbd "M-è") "[")
(global-set-key (kbd "M-+") "]")
\end{verbatim}
\subsection{Insert "\{" and "\}"}
\label{sec:org02740f1}
On italian keyboards the curly parentheses are inserted with 'Alt+é' or 'Alt+*'. This command makes it possible to do so.
\begin{verbatim}
(global-set-key (kbd "M-é") "{")
(global-set-key (kbd "M-*") "}")
\end{verbatim}
\section{Misc}
\label{sec:orgbbe5b00}

\begin{verbatim}
  ;; (defun dashboard-refresh-buffer ()
  ;;   (when (get-buffer dashboard-buffer-name)
  ;;     (kill-buffer dashboard-buffer-name))
  ;;   (dashboard-insert-startupify-lists)
  ;;   (switch-to-buffer dashboard-buffer-name))
(load-file "~/.emacs.d/elegant-emacs/elegance.el")
\end{verbatim}
\end{document}
